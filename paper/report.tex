% report.tex
% Final Report for MIT 6.858 
% Jeff Stewart and Veer Dedhia

\documentclass[journal]{IEEEtran}
\usepackage{cite}

\begin{document}
%
\title{ROP Compiler}

\author{Jeff~Stewart,
        Veer~Dedhia}


% The paper headers
\markboth{MIT 6.858}{}

% make the title area
\maketitle

\section{Introduction}
Here is an introduction. Here is an introduction. Here is an introduction. Here is an introduction. Here is an introduction.
Here is an introduction. Here is an introduction. Here is an introduction. Here is an introduction. Here is an introduction.
Here is an introduction. Here is an introduction. Here is an introduction. Here is an introduction. Here is an introduction.
Here is an introduction. Here is an introduction. Here is an introduction. Here is an introduction. Here is an introduction.

Here is where we put Motivation.
  - why a rop compiler is needed

\subsection{Background}
  - what is ROP (why it matters)
  - why existing ones aren't too usable
      - other existing tools, and their problems

\section{Implementation Details}
    - based off of Q \cite{schwartz2011q}
    a) Design
        1) gadget finder
        2) goal language
        3) gadget scheduler
          - How our compiler interacts with existing defenses (ASLR, w\^x)

\section{Example Usage}
  - bof.py
  - rsync

\section{Future Work}
  - Synthesizing new gadgets by combining ones

\bibliographystyle{plain}
\bibliography{sources.bib}{}

\end{document}
